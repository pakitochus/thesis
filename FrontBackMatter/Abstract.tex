%*******************************************************
% Abstract
%*******************************************************
%\renewcommand{\abstractname}{Abstract}
\pdfbookmark[1]{Abstract}{Abstract}
\begingroup
\let\clearpage\relax
\let\cleardoublepage\relax
\let\cleardoublepage\relax

\chapter*{Abstract}
The rise of neuroimaging in the last years has provided physicians and radiologist with the ability to study the brain with unprecedented ease. This led to a new biological perspective in the study of neurodegenerative diseases, allowing the characterization of different anatomical and functional patterns associated with them. Computer Aided Diagnostic (CAD) systems use statistical techniques for preparing, processing and extracting information from neuroimaging data pursuing a major goal: optimize the process of analysis and diagnosis of neurodegenerative diseases and mental conditions.

With this thesis we focus on three different stages of the CAD pipeline: preprocessing, feature extraction and validation. For preprocessing, we have developed a method that target a relatively recent concern: the confounding effect of false positives due to differences in the acquisition at multiple sites. Our method can effectively merge datasets while reducing the acquisition site effects. Regarding feature extraction, we have studied decomposition algorithms (independent component analysis, factor analysis), texture features and a complete framework called Spherical Brain Mapping, that reduces the 3-dimensional brain images to two-dimensional statistical maps. This allowed us to improve the performance of automatic systems for detecting Alzheimer's and Parkinson's diseases. Finally, we developed a brain simulation technique that can be used to validate new functional datasets as well as for educational purposes. 

Guide: 
\begin{center}
	\url{https://plg.uwaterloo.ca/~migod/research/beckOOPSLA.html}
\end{center}

\begin{otherlanguage}{spanish}
\pdfbookmark[1]{Resumen}{Resumen}
\chapter*{Resumen}
Resumen de la tesis en español. 
\end{otherlanguage}

\endgroup			

\vfill
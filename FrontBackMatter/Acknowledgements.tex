% !TEX TS-program = pdflatex
% !TEX root = ../Tesis.tex

%*******************************************************
% Acknowledgements
%*******************************************************
\pdfbookmark{Agradecimientos}{Agradecimientos}

\begin{otherlanguage}{spanish}
\chapter*{Agradecimientos}
\textit{El trabajo que aquí se presenta se ha financiado mediante un contrato de investigación asociado al proyecto de excelencia P11-TIC- 7103 de la Consejer\'ia de Econom\'ia, Innovaci\'on, Ciencia y Empleo (Junta de Andaluc\'ia, Spain). Para la estancia en la Universidad de Cambridge se ha contado con una financiación complementaria del Vicerrectorado de Relaciones Internacionales de la Universidad de Granada y el CEI BioTic, bajo el programa Convocatoria de Movilidad Internacional de Estudiantes de Doctorado Curso 2013/2014. Asimismo, a lo largo de este periodo predoctoral se ha contado con financiación complementaria de los proyectos P09-TIC-4530 de la Junta de Andalucía y TEC2008-02113, TEC2012-34306 y TEC2015-64718-R del MINECO/FEDER.}
\bigskip

Esta tesis no sería posible de no haber contado con Juanma y Javi como directores del proyecto. A vosotros os debo la máxima gratitud por vuestra calidad personal y profesional. Porque me habéis guiado desde el primer día hasta el último, a la vez que me habéis dejado una grandísima libertad en mi trabajo. Hemos recorrido todo este camino juntos, y me llevo eso que llaman una formación `integral', intelectual, laboral y en valores. No podría imaginar mejores directores para esta tesis. Gracias por todo este camino, y todo lo que nos queda. 

Al resto del grupo SiPBA y colaboradores; a Diego Salas, que fue el que me trajo a este grupo, por esas cosas de conocerse con una guitarra delante; Fermín, Ignacio, Diego Castillo (novatillo!), Andrés, Brahim, Laila, Miriam, Rosa, Alberto... Gracias por dar vida a este grupo de investigación, donde se respira ciencia y buen hacer. 

A John Suckling, por acogerme como uno más -y, si hay suerte, en próximas ocasiones- y a todos los compañeros del departamento de Psiquiatría y el departamento del Regal en Cambridge. Gracias porque allí se demuestra que multidisciplinaridad es un palabro que solo significa `tenemos mucho que aprender unos de otros'. 

Al departamento de Teoría de la Señal, Telemática y Comunicaciones, donde he aprendido muchas cosas y donde he tenido el gusto de descubrirme en la docencia, un trabajo que al final me ha encantado. Y a la Universidad de Granada, donde me he formado todos estos años, y donde miles de personas trabajan para que siga siendo una de las referencias a nivel español e internacional. 

A todos los compis del CITIC, muy especialmente a mis compañeros de despacho, donde al final hemos hecho piña y estamos como en casa: Fermín, Pablo, Rober, Marta, Gabri, Diego y Samuel. 

A mis padres, que han estado apoyándome desde el primer día, incluso cuando no había muchas perspectivas de financiación. Sois un ejemplo de trabajo y dedicación, y gracias a vuestro tesón estoy escribiendo estas líneas. Siempre recordaré vuestro ``nosotros no tenemos tierras, ni negocios, lo único que podemos daros es una educación''. No es tangible, y quizá por eso, es el mejor patrimonio que se puede heredar. Gracias por todo lo que me habéis transmitido, cuidado y dado a lo largo de toda mi vida. 

Gracias a mi familia, y a todos los amigos que me habéis aguantado en esos: ``no puedo, que mañana hay que levantarse temprano para la tesis''; que me habéis preguntado, os habéis preocupado por qué tal andamos. Sin duda, esto también es mérito vuestro. A Jiménez y Rocío, que este es vuestro año, a Elena, compi Teleco.. A todos los que participásteis en la encuesta para decidir el mapa de color de las imágenes médicas, sin duda el impacto de vuestros votos os garantiza una presencia en este documento. 

A la banda sonora de mi vida: los muy grandes Adversos, el Coro de Ciencias, el de Consolación y los maravillosos componentes de Musicalarte. Y a la música en general, ese virus que me infecta día tras día, el lenguaje de lo inexpresable. El árbol de la vida cuyo fruto fue, casualmente, mi interés por las señales, sin el cual no habría trabajado en esta tesis. 

A los compañeros del Three-minute thesis, por todos esos momentos que hemos vivido juntos, formación, entrevistas, ensayos y en definitiva, todo lo que hemos crecido gracias a esta aventura. Y a Carlos, Susana y Emilio, por guiarnos en ese camino. 

A tí, que estás leyendo estas líneas, por interesarte por nuestra humilde aportación a la ciencia. Esperamos que sea de tu agrado y que la disfrutes una mínima parte de todo lo que la hemos disfrutado nosotros. 

Gracias a tí, Inma. Por todo lo que hemos pasado juntos, por querer compartir una vida conmigo. Porque día tras día me demuestras que el tesón y la fuerza que tienes es única. Porque pese a tu dulzura y tu aparente fragilidad, sigues resistiendo a todas las adversidades, luchando por lo que crees. Porque eres muy grande, mucho más de lo que crees, y de lo que algunos ciegos son incapaces de percibir. Tú eres la luz que veo cada mañana y cada noche. El potencial de mi energía. Mi hogar, al que siempre quiero volver. Gracias.

\end{otherlanguage}
%\begin{flushright}
%\itshape
%We have seen that computer programming is an art, \\
%because it applies accumulated knowledge to the world, \\
%because it requires skill and ingenuity, \\
%and especially because it produces objects of beauty. \\
%\medskip
%--- Donald Ervin Knuth
%\end{flushright}
%
%\bigskip
%\bigskip

%\heartpar{I wish first of all to thank the members of the Italian \TeX{} and \LaTeX{} User Group, in particular Claudio Beccari, Marco Brunero, Fabiano Busdraghi, Gustavo Cevolani, Rosaria D'Addazio, Massimiliano Dominici, Gloria Faccanoni, Daniele Ferone, Tommaso Gordini, Gianluca Gorni, Enrico Gregorio, Maurizio Himmelmann, Jer\'onimo Leal, Paride Legovini, Lapo Filippo Mori, Andrea Tonelli, Ivan Valbusa, Emiliano Giovanni Vavassori and Emanuele Vicentini, for their invaluable aid during the writing of this work, the detailed explanations, the patience and the precision in the suggestions, the supplied solutions, the competence and the kindness: thank you, guys!
%Thanks also to all the people who have discussed with me on the forum of the Group, prodigal of precious observations and good advices.
%Finally, thanks to Andr\'e Miede, for his wonderful ClassicThesis style, and to Daniel Gottschlag, who gave to me the hint for this original reworking.}


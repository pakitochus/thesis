%********************************************************************
% Appendix
%*******************************************************
% If problems with the headers: get headings in appendix etc. right
%\markboth{\spacedlowsmallcaps{Appendix}}{\spacedlowsmallcaps{Appendix}}
\chapter{Datasets}\label{ch:datasets}
Seven datasets have been used in this thesis, covering three imaging modalities and three different disorders. A summary of these can be found on Table~\ref{tab:datasetsOverview}, followed by a longer description of each one.
\begin{table}[h]
	\myfloatalign
	\begin{tabular}{lllll} \toprule
		\tableheadline{Acronym} & \tableheadline{Entity}
		& \tableheadline{Disease} & \tableheadline{Modality}
		& \tableheadline{Drug} \\ \midrule
		ADNI-MRI & \ac{ADNI} & \ac{AD} &  \ac{MRI} & - \\
		AIMS-MRI & \ac{MRC-AIMS} & \ac{ASD} & \ac{MRI} & - \\
		%postulant quo & westeuropee & sanctificatec \\
		\midrule
		ADNI-PET & \acs{ADNI} & \ac{AD} & \ac{PET} & FDG \\
		\midrule
		VDLN-HMPAO & \acs{VDLN} & \ac{AD} & \ac{SPECT} & HMPAO \\
		VDLN-DAT & \acs{VDLN} & \ac{PKS} & \ac{SPECT} & DaTSCAN \\
		VDLV-DAT & \acs{VDLV} & \ac{PKS} & \ac{SPECT} & DaTSCAN \\
		PPMI-DAT & \acs{PPMI} & \ac{PKS} & \ac{SPECT} & DaTSCAN \\
		\bottomrule
	\end{tabular}
	\caption[Summary of the datasets used in this thesis.]{Summary of the datasets used in this thesis.}
	\label{tab:datasetsOverview}
\end{table}
\section{\acs{MRI}}
\subsection{ADNI-MRI and the Alzheimer's Disease Neuroimaging Initiative}\label{sec:adnimri}
The \acf{ADNI} was launched in 2003 as a public-private partnership, led by Principal Investigator Michael W. Weiner, MD. Its primary goal has been to test whether serial \ac{MRI}, \ac{PET}, other biological markers, and clinical and neuropsychological assessment can be combined to measure the progression of \ac{MCI} and early \ac{AD}. Determination of sensitive and specific markers of very early AD progression is intended to aid researchers and clinicians to develop new treatments and monitor their effectiveness, as well as lessen the time and cost of clinical trials.

\ac{ADNI} is the result of efforts of many co-investigators from a broad range of academic institutions and private corporations, and up to 1500 adults (ages 55 to 90) were recruited from over 50 sites across the U.S. and Canada in \ac{ADNI} and its following initiatives \ac{ADNI}-GO and \ac{ADNI}-2. Subjects had completed at least 6 years of education, and were fluent in Spanish or English. For up-to-date information on inclusion/exclusion criteria and other topics, see \url{www.adni-info.org}.

In this thesis we will use data belonging to the \ac{ADNI}-1 initiative. In particular, the database that we call ADNI-MRI correspond to the \ac{MRI} volumes from the `ADNI1: Screening 1.5T' collection (subjects who have a screening data). It contains 818 T1-weighted \ac{MRI} images from \ac{CTL} subjects (229), \ac{MCI} (398) and \ac{AD} (191) (see demographic details at Table~\ref{tab:demoADNI-MRI}). To avoid prevalence in some of our experiments, we have created two subsets that will be used throughout this thesis, randomly selecting 180 subjects from the \ac{AD} and \ac{CTL} classes.

\begin{table}[h]
	\myfloatalign
	\begin{tabular}{lllcc} 
		\toprule
		\tableheadline{Group} & \tableheadline{Sex} & \tableheadline{N} & \tableheadline{Age ($\mu \pm \sigma$ years)} & \tableheadline{MMSE ($\mu \pm \sigma $)}\\
		\midrule
		\multirow{2}{*}{AD} & F & 91 & $74.75 \pm 7.63$ & $22.98 \pm 2.65$ \\
							& M & 100 & $75.72\pm 2.35$ &	$23.15\pm 2.35$\\\midrule
		\multirow{2}{*}{CTL} & F & 110 & $76.04\pm 0.92$ & $29.21\pm 0.92$\\
							& M & 119 & $75.70 \pm 1.03$ & $28.95\pm 1.03$\\\midrule
		\multirow{2}{*}{MCI} &F & 141 & $73.66\pm 2.32$ &	$26.73\pm 2.32$\\
							& M & 257 & $75.31 \pm 2.11$ &$26.97\pm 2.11$\\
		\bottomrule
	\end{tabular}
	\caption[Demographics of the ADNI-MRI dataset.]{Demographics of the ADNI-MRI dataset.}
	\label{tab:demoADNI-MRI}
\end{table}

 
Depending on the experiment, we may use spatially normalized (or registered) T1-weighted \ac{MRI} images, using the SPM8 software (see Section~\ref{sec:spatial}) \cite{spm_book}, or segmented \ac{GM} and \ac{WM} maps. Segmentation was performed using the VBM8 toolbox for SPM \cite{vbm_ref}. 

\subsection{AIMS-MRI, \acs{MRC-AIMS} Consortium}\label{sec:aims-mri}
The \acs{MRC-AIMS} consortium created this database to study \ac{ASD}. A number of adult, right-handed males with no significant mean differences in age and full-scale IQ were recruited by advertisement. Participants were excluded from the study if they had a history of major psychiatric disorder or medical illness affecting brain function (e.g. psychosis or epilepsy), or current drug misuse (including alcohol), or were taking antipsychotic medication, mood stabilizers or benzodiazepines. 

All participants with \ac{ASD} were diagnosed according to International Classification of Diseases, 10th Revision (ICD-10) research criteria, and confirmed using the Autism Diagnostic Interview-Revised (ADI-R) \cite{Lord1994}. Autism Diagnostic Observation Schedule (ADOS) \cite{Lord2000} was performed, but the score was not considered as an inclusion criteria. \ac{ASD} participants, to be included, must have scored above the ADI-R cut-off in the three domains of impaired reciprocal social interaction, communication and repetitive behaviours and stereotyped patterns, although failure to reach cut-off in one of the domains by one point was permitted. Intellectual ability was assessed using the Wechsler Abbreviated Scale of Intelligence (WASI) \cite{Wechsler1999a}, ensuring the participants fell within the high-functioning range on the spectrum defined by a full-scale IQ > 70.

In this work, only structural \ac{MRI} from participants recruited at the Institute of Psychiatry, King’s College London (LON) and the Autism Research Centre, University of Cambridge (CAM) were included, where an equivalent set of images were acquired from each participant. This makes for a total number of 136 adult, right-handed males (68 with \ac{ASD} and 68 matched controls). The demographics of the participants are shown in detail in Table~\ref{tab:demoMRCAIMS}. 

\begin{table}[h]
	\myfloatalign
	\begin{tabular}{lllcc} 
		\toprule
		\tableheadline{Database} & \tableheadline{Group} & \tableheadline{N} & \tableheadline{Age ($\mu \pm \sigma$ years)} & \tableheadline{IQ ($\mu \pm \sigma $)}\\
		\midrule
		\multirow{2}{*}{LON} & \ac{ASD} & 39 & $28.74 \pm 6.52$ & $111.28 \pm 13.13$ \\
		& \ac{CTL} & 40 & $25.30\pm6.62$ &	$104.67\pm11.16$\\\midrule
		\multirow{2}{*}{CAM} & \ac{ASD} & 29 & $26.83\pm4.64$ & $115.83\pm11.88$\\
		& \ac{CTL} & 28 & $26.75 \pm 7.32$ & $115.25\pm13.67$\\\midrule
		\multirow{2}{*}{ALL} &\ac{ASD} & 68 & $25.90\pm6.95$ &	$109.03\pm13.31$\\
		& \ac{CTL} & 68 & $27.93 \pm 5.87$ &$113.22\pm12.81$\\
		\bottomrule
	\end{tabular}
	\caption[Demographics of the AIMS-MRI dataset.]{Demographics of the AIMS-MRI dataset.}
	\label{tab:demoMRCAIMS}
\end{table}

Structural \ac{MRI} were obtained using Driven Equilibrium Single Pulse Observation of T1 and T2 (DESPOT1, DESPOT2) \cite{deoni2008standardized} at King’s College London and University of Cambridge, both with 3T GE Medical Systems HDx scanners. Using multiple Spoilt Gradient Recall (SPGR) acquisitions in the DESPOT1 sequence and Steady State Free Procession (SSPF) acquisitions in the DESPOT2 sequence, with different flip angles and repetition times, \ac{qT1} and \ac{qT2} maps were calculated with a custom ImageJ plug-in package. Correction of main and transmit magnetic field (B0 and B1) inhomogeneity effects was performed during the estimation of T1 and T2.

For accurate registration to the standard stereotatic space of the \ac{MNI}, a \ac{synT1} images were created based on the \ac{qT1} maps \cite{Ecker2013,Ecker2012,Lai2012}. The \ac{synT1} images were then segmented using New Segment into \ac{GM} and \ac{WM} maps, and normalized to the \ac{MNI} space using DARTEL in SPM8 \cite{spm_book}, with modulation (preserve volume) to retain information of regional/local \ac{GM} and \ac{WM} volumes, and smoothed with a 3mm FWHM Gaussian Kernel to account for inter-subject mis-registration. The \ac{synT1}, \ac{qT1} and \ac{qT2} maps were also registered to the standard \ac{MNI} space using the same DARTEL flow fields, but without modulation (preserve concentration) to retain information of regional/local T1 contrast, T1 relaxation time, and T2 relaxation times, and smoothed with a 3mm FWHM Gaussian kernel. Therefore, there were five different modalities: \ac{qT1}, \ac{qT2}, \ac{synT1} map, \ac{GM} and \ac{WM} maps, for each every participant, which allows us to observe the impact of our \ac{SWPCA} correction of site-related undesired variance on quantitative (\ac{qT1} and \ac{qT2}), simulated (\ac{synT1}) images and probability maps (\ac{GM} and \ac{WM}).
	
During the pre-processing of the images, several procedures targeted the reduction of inter-subject and inter-site geometric distortion, amongst them the correction of B0 and B1 field inhomogeneity effects and the registration to \ac{MNI} space. Many other algorithms have been proposed to help in this task. However, the study of their relative performance lies beyond the scope of this article. Following image registration, it was assumed that only the intensity of the maps was affected between sites.

\section{\acs{PET}}
\subsection{ADNI-PET, Alzheimer's Disease Neuroimaging Initiative}\label{sec:adnipet}
The ADNI-PET database also refers from images acquired at the \ac{ADNI}. However, in this case we will use $^{18}F$-FDG \ac{PET} images, used to estimate the metabolic activity of the brain. This radiopharmaecutical is a glucose analog, and its distribution of the brain can be used to trace glucose metabolism, and by extension, brain function. The acquisition procedure is detailed at their website (\url{http://www.loni.ucla.edu/ADNI/Data/ADNI_Data.shtml}). In brief, \ac{PET} images were acquired at a variety of scanners nationwide using either a 30-min six frame scan acquired 30–60 min post-injection or a static 30-min single-frame scan acquired 30–60 min post-injection. The resulting images were aligned along the AC-PC line and a subject-specific intensity normalization was applied, so that the sum of all voxels in the cerebral mask of each subject summed to one. Finally, images were smoothed using the smallest resolution across all scanners, with a uniform isotropic filter of 8mm FWHM. 

We used 403 images from the ADNI1 screening database: 95 \ac{PET} images from \ac{AD} affected subjects,  207 images from \ac{MCI} affected subjects and 101 images from \ac{CTL}. Demographic details of this population can be found at Table~\ref{tab:demoADNI-PET}. 

\begin{table}[h]
	\myfloatalign
	\begin{tabular}{lllcc} 
		\toprule
		 \tableheadline{Group} & \tableheadline{N} & \tableheadline{Age ($\mu \pm \sigma$ years)} & \tableheadline{MMSE ($\mu \pm \sigma $)}\\
		\midrule
		 \ac{AD} & 95 & $77.2 \pm 7.5$ & $23.4 \pm 2.1$ \\
		 \ac{MCI} & 207 & $76.6 \pm 7.2$ &	$27.2 \pm 1.7$\\
		\ac{CTL} & 101 & $77.8 \pm 4.6$ & $29.0 \pm 1.1$\\
		\bottomrule
	\end{tabular}
	\caption[Demographics of the ADNI-PET dataset.]{Demographics of the ADNI-PET dataset.}
	\label{tab:demoADNI-PET}
\end{table}

Further preprocessing of these images included spatial normalization, via the SPM8 software and its \ac{PET} template, and intensity normalization. The type of intensity normalization will be specified in each chapter, because it depends on the application. 

\section{\acs{SPECT}}

\subsection{VDLN-HMPAO, Virgen de las Nieves}\label{sec:vdlnhmpao}
The database is built up of imaging studies of subjects following the protocol of a hospital-based service. First, the neurologist evaluated the cognitive function, and those patients with findings of memory loss or dementia were referred to the nuclear medicine department in the \acf{VDLN} (Granada, Spain), in order to acquire complementary screening information for diagnosis\footnote{Clinical information is unfortunately not available for privacy reasons, but only demographic  information}. 

The images were visually evaluated by experienced physicians, using 4 different labels: \ac{CTL} for subjects without scintigraphic abnormalities and mild perfusion deficit (\ac{AD}-1), moderate deficit (\ac{AD}-2) and severe deficit (\ac{AD}-3), to differentiate between levels of hypo-perfusion patterns compatible with \ac{AD}. 

In total, the database consists of 97 subjects: 41 \ac{CTL}, 30 \ac{AD}-1, 22 \ac{AD}-2 and 4 \ac{AD}-3 (see table \ref{tab:demoVDLN-HMPAO}for demographic details). Since the patients are not pathologically confirmed, the subject's labels possesses some degree of uncertainty, as the pattern of hypo-perfusion may not reflect the underlying pathology of AD, nor the different classification of scans necessarily reflect the severity of the patients symptoms. However, when pathological information is available, visual assessments by experts have been shown to be very sensitive and specific labelling methods, in contrast to neuropsychological tests \cite{jobst_accurate_1998,dougall_systematic_2004}. Given that this is an inherent limitation of `in vivo' studies, our working-assumption is that the labels are true, considering the subject label positive when belonging to any of the \ac{AD} classes, and negative otherwise. 

\begin{table}
	\begin{center}
		\begin{tabular}{lccc}
			\toprule
			          & \tableheadline{N} & \tableheadline{Sex(M/F)(\%)} & \tableheadline{Age ($\mu \pm \sigma$ years)} \\ 
			          \midrule
			\ac{CTL}  &     41      & 32.95/12.19  &     $71.51 \pm 7.99$ \\
			\ac{AD}-1 &     29      & 10.97/18.29  &    $65.29 \pm 13.36$ \\
			\ac{AD}-2 &     22      &  13.41/9.76  &     $65.73 \pm 8.25$ \\
			\ac{AD}-3 &      4      &    0/2.43    &        $76 \pm 9.90$ \\ 
			\bottomrule
		\end{tabular}
		\caption[Demographic details of the VDLN-HMPAO dataset.]{Demographic details of the VDLN-HMPAO dataset. \ac{CTL} = Normal Controls, \ac{AD}-1 = possible \ac{AD}, \ac{AD}-2 = probable \ac{AD}, \ac{AD}-3 = certain \ac{AD}. $\mu$ and $\sigma$ stands for population mean and standard deviation respectively.}
		\label{tab:demoVDLN-HMPAO}
	\end{center}
\end{table}

Images were subsequently registered to a custom \ac{SPECT} template, and a posterior intensity normalization could have been applied, which will be reported in each particular case. 

\subsection{VDLN-DAT, Virgen de las Nieves}\label{sec:vdlndat}
This database was supplied by the \ac{VDLN} (Granada). It contains patients that were derived to the nuclear medicine service between 2007 and 2012, after a \ac{PKS} clinical diagnosis, to perform a confirmatory DaTSCAN analysis of the nigrostriatal system. \ac{SPECT} images were acquired after the injection of 185 MBq of DATSCAN on previously thyroid-blocked patients, and acquired using a three head Picker Prism 3000 gamma camera. 

Labels were established by both visual interpretation and exploration quantification, using a delimitation of \ac{ROI} in the striatum. The visual interpretation was subjective by the physicians, mainly by analyzing the total intensity in both striatum, possible asymmetries between them and the existent relationship between the striatal and the background brain activity. For its part, the exploration quantification was performed by computing the radiopharmaceutical activity in counts per pixel for each area, from which measures such as total and differential activity between the striatum and the occipital lobe were obtained. 

This database contains 148 subjects: 45 \ac{CTL}, 73\ac{PD} and 30 \ac{SWEDD}, from which only \ac{CTL} and \ac{PD} were used. For more details on the demographics of this dataset, see Table~\ref{tab:demoVDLN-DAT}. 

\begin{table}[h]
	\myfloatalign
	\begin{tabular}{lllc} 
		\toprule
		\tableheadline{Group} & \tableheadline{Sex} & \tableheadline{N} & \tableheadline{Age ($\mu \pm \sigma$ years)}\\
		\midrule
		\multirow{2}{*}{\ac{CTL}}   & F & 20 & $73.20 \pm 9.24$	\\
		& M & 25 & $69.24 \pm 10.80$  \\
		\midrule
		\multirow{2}{*}{\ac{SWEDD}} & F & 14 & $73.16 \pm 4.83$\\
		& M & 16 & $71.50 \pm 8.13$	\\
		\midrule
		\multirow{2}{*}{\ac{PD}}    & F & 28 & $69.57 \pm 8.81$	\\
		& M & 45 & $69.34 \pm 9.38$	\\
		\bottomrule
	\end{tabular}
	\caption[Demographics of the VDLN-DAT dataset.]{Demographics of the VDLN-DAT dataset.}
	\label{tab:demoVDLN-DAT}
\end{table}

After the image acquisition, images were preprocessed by means of a spatial normalization, using a template developed by the SiPBA Research Group \cite{Salas-Gonzalez2015}. A posterior intensity normalization could have been performed, depending on the experiment, and it has specifically been stated in each chapter. 

\subsection{VDLV-DAT, Virgen de la Victoria Hospital}\label{sec:vdlvdat}
The images were obtained after a period of between 3 and 4 hours after the intravenous injection of 185 MBq (5 mCi) of DaTSCAN, with prior thyroid blocking with Lugol's solution. The tomographic study (\ac{SPECT}) with Ioflupane/FP-CIT-I-123 was performed using a General Electric gamma camera, Millennium model, equipped with a dual head and general purpose collimator. A 360-degree circular orbit was made around the cranium, at 3-degree intervals, 60 images with a duration of 35 seconds per interval, $128\times128$ matrix. Image reconstruction was carried out using filtered back-projection algorithms without attenuation correction \cite{Shepp82,Vardi1985}, application of a Hanning filter (frequency 0.7) and images were obtained with transaxial cuts, following the method proposed in \cite{Ramirez2009}. 

The images were interpreted by three Nuclear Medicine specialists, with masking of the clinical orientation. Visual assessment was established by exclusively considering the normal/abnormal criterion and after arriving at a consensus report between the three specialists, \ie whether the FP-CIT \ac{SPECT} allowed differentiation of a group of conditions with presynaptic involvement from others in which their integrity is assumed, without trying to assign them to different clinical groups within the set of pathological studies. A study was considered to be normal when bilateral, symmetrical uptake appeared in caudate and putamen nuclei, and abnormal when there were areas of qualitatively reduced uptake in any of the striatal structures. 

\begin{table}[h]
	\myfloatalign
	\begin{tabular}{lllc} 
		\toprule
		\tableheadline{Group} & \tableheadline{Sex} & \tableheadline{N} & \tableheadline{Age ($\mu \pm \sigma$ years)}\\
		\midrule
		\multirow{2}{*}{\ac{CTL}}   & F & 54 & $68.51 \pm 10.54$	\\
		& M & 54 & $69.58 \pm 10.01$  \\
		\midrule
		\multirow{2}{*}{\ac{PD}}    & F & 47 & $67.61 \pm 10.24$	\\
		& M & 53 & $69.52 \pm 8.78$	\\
		\bottomrule
	\end{tabular}
	\caption[Demographics of the VDLV-DAT dataset.]{Demographics of the VDLV-DAT dataset.}
	\label{tab:demoVDLV-DAT}
\end{table}

A total of 208 subjects (100 \ac{PD} and 108 \ac{CTL}), randomly selected from the total studies performed in this center until December 2008 and referred to it because of a movement disorder, were included in the study. Clinical diagnosis, a parameter used as `gold Standard' to establish the existence of \ac{PKS}, was made using the diagnostic criteria established previously, with an established minimum follow-up period of 18 months. Those patients who were receiving treatment with drugs that had known or suspected effect on the level of the dopaminergic transporters through direct competitive mechanism were excluded. A more detailed description of the database can be found in \cite{Lozano2007}.

Images were registered to our custom DaTSCAN template, and then, an intensity normalization was usually applied (which will be specified at each experiment). 

\subsection{PPMI-DAT, Parkinson's Progression Markers Initiative}\label{sec:ppmi}
\ac{PPMI} --a public-private partnership-- is funded by The Michael J. Fox Foundation for Parkinson's Research and funding partners, including Abbott, Biogen Idec, F. Hoffman-La Roche Ltd., GE Healthcare, Genentech and Pfizer Inc. It is a landmark study launched in 2010 aimed at finding biomarkers for \ac{PD} diagnosis and treatments. 

The images in this dataset were acquired after the injection of between 111 and 185 MBq of DaTSCAN, in subjects that had been pretreated with saturated iodine solution. To facilitate image processing and preserve lateality, a $^{57}$Co line marker was affixed along the canthomeatal line, which does not affect the \acp{ROI} \cite{PPMI,Inititative2010}. For up-to-date information on the study, visit \url{www.ppmi-info.org}.

The PPMI-DAT refers to a subset of the DaTSCAN screening data of the \ac{PPMI} initiative, containing 301 subjects: 111 \ac{CTL}, 31 \ac{SWEDD} and 159 \ac{PD}. These images were later registered to the custom DaTSCAN template defined previously, and depending on the experiment, an intensity normalization was applied. Demographic details of these subjects can be found at Table~\ref{tab:demoPPMI-DAT}.

\begin{table}[h]
	\myfloatalign
	\begin{tabular}{lllc} 
		\toprule
		\tableheadline{Group} & \tableheadline{Sex} & \tableheadline{N} & \tableheadline{Age ($\mu \pm \sigma$ years)}\\
		\midrule
		\multirow{2}{*}{\ac{CTL}}   & F & 45 & $55.37 \pm 10.97$	\\
								    & M & 66 & $59.68 \pm 11.48$  \\
		\midrule
		\multirow{2}{*}{\ac{SWEDD}} & F & 15 & $56.26 \pm 10.25$\\
									& M & 16 & $60.18 \pm 11.46$	\\
		\midrule
		\multirow{2}{*}{\ac{PD}}    & F & 45 & $61.20 \pm 10.18$	\\
								    & M & 114 & $62.94 \pm 8.70$	\\
		\bottomrule
	\end{tabular}
	\caption[Demographics of the PPMI-DAT dataset.]{Demographics of the PPMI-DAT dataset.}
	\label{tab:demoPPMI-DAT}
\end{table}
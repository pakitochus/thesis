%********************************************************************
% Appendix
%*******************************************************
% If problems with the headers: get headings in appendix etc. right
%\markboth{\spacedlowsmallcaps{Appendix}}{\spacedlowsmallcaps{Appendix}}
\chapter{Datasets}\label{ch:datasets}
Many dataset are used in this thesis, covering three imaging modalities and three disorders. A summary of these can be found on Table~\ref{tab:datasetsOverview}, folowed by a longer description of each one.
\begin{table}[h]
	\myfloatalign
	\begin{tabularx}{\textwidth}{lllll} \toprule
		\tableheadline{Acronym} & \tableheadline{Origin}
		& \tableheadline{Disease} & \tableheadline{Modality}
		& \tableheadline{Drug} \\ \midrule
		ADNI-MRI & \ac{ADNI} & \ac{AD} &  \ac{MRI} & - \\
		AIMS-MRI & \ac{MRC-AIMS} & \ac{ASD} & \ac{MRI} & - \\
		%postulant quo & westeuropee & sanctificatec \\
		\midrule
		ADNI-PET & \ac{ADNI} & \ac{AD} & \ac{PET} & \ac{HMPAO} \\
		\midrule
		VDLN-HMPAO & \ac{VDLN} & \ac{AD} & \ac{SPECT} & \ac{HMPAO} \\
		VDLN-DAT & \ac{VDLN} & \ac{PKS} & \ac{SPECT} & DaTSCAN \\
		VDLV-DAT & \ac{VDLV} & \ac{PKS} & \ac{SPECT} & DaTSCAN \\
		PPMI-DAT & \ac{PPMI} & \ac{PKS} & \ac{SPECT} & DaTSCAN \\
		\bottomrule
	\end{tabularx}
	\caption[Summary of the datasets used in this thesis.]{Summary of the datasets used in this thesis.}
	\label{tab:datasetsOverview}
\end{table}
\section{Magnetic Resonance Imaging}
\subsection{ADNI-MRI, Alzheimer's Disease Neuroimaging Initiative}\label{sec:adnimri}
AD

\subsubsection{Baseline - VAF}\label{sec:baseline}
In order to establish a baseline to assess the predictive ability of our maps, we will use the \acf{VAF} paradigm, described in \cite{Stoeckel04}. This approach uses the whole 3D \ac{GM} or \ac{WM} segmented MR images and then uses all voxels of the 3D images as features in the SVM classification, yielding the performance values shown in Table~\ref{tab:perfVAF}. The performance of the \ac{SBM} maps will be compared to these. 

\begin{table*}[htp]
	\myfloatalign
	\begin{tabularx}{\textwidth}{Xccc}
		\tableheadline{Approach}  & \tableheadline{Accuracy} & \tableheadline{Sensitivity} & \tableheadline{Specificity}\\
		\midrule
		\ac{VAF} (\ac{GM})  & $0.768 \pm 0.011$ & $0.752 \pm 0.016$ & $0.785 \pm 0.016$ \\
		\ac{VAF} (\ac{WM})  & $0.642 \pm 0.009$ & $0.668 \pm 0.012$ & $0.617 \pm 0.013$ \\

		\bottomrule
	\end{tabularx}
	\caption{Performance values (Average $\pm$ Standard Deviation) for the  Voxels as Features approach in both \ac{GM} and \ac{WM} tissues.\label{tab:perfVAF}}
\end{table*}


\subsection{AIMS-MRI, MRC-AIMS Consortium}\label{sec:aims-mri}
Structural MRI were analysed from 136 adult, right-handed males (68 with ASD and 68 matched controls) with no significant mean differences in age and full-scale IQ, acquired from the centres contributing to the UK Medical Research Council Autism Imaging Multi-centre Study (MRC AIMS) (Ecker, et al., 2013; Ecker, et al., 2012) and recruited by advertisement. In this work, only participants recruited at the Institute of Psychiatry, King’s College London (LON) and the Autism Research Centre, University of Cambridge (CAM) were included where an equivalent set of images were acquired from each participant. 

Participants were excluded from the study if they had a history of major psychiatric disorder or medical illness affecting brain function (e.g. psychosis or epilepsy), or current drug misuse (including alcohol), or were taking antipsychotic medication, mood stabilizers or benzodiazepines. 

All participants with ASD were diagnosed according to International Classification of Diseases, 10th Revision (ICD-10) research criteria, and confirmed using the Autism Diagnostic Interview-Revised (ADI-R) (Lord, et al., 1994). Autism Diagnostic Observation Schedule (ADOS) (Lord, et al., 2000) was performed, but the score was not considered as an inclusion criteria. ASD participants, to be included, must have scored above the ADI-R cut-off in the three domains of impaired reciprocal social interaction, communication and repetitive behaviours and stereotyped patterns, although failure to reach cut-off in one of the domains by one point was permitted. Intellectual ability was assessed using the Wechsler Abbreviated Scale of Intelligence (WASI) (Wechsler, 1999), ensuring the participants fell within the high-functioning range on the spectrum defined by a full-scale IQ > 70. The demographics of the participants are shown in detail in  Table~\ref{tab:demoMRCAIMS}. 
\begin{table}[h]
	\myfloatalign
	\begin{tabularx}{\textwidth}{lllXX} 
		\toprule
		\tableheadline{Database} & \tableheadline{Group} & \tableheadline{N} & \tableheadline{Age ($\mu \pm \sigma$ years)} & \tableheadline{IQ ($\mu \pm \sigma $)}\\
		\midrule
		\multirow{2}{*}{LON} & ASD & 39 & $28.74 \pm 6.52$ & $111.28 \pm 13.13$ \\
		& CTL & 40 & $25.30\pm6.62$ &	$104.67\pm11.16$\\\midrule
		\multirow{2}{*}{CAM} & ASD & 29 & $26.83\pm4.64$ & $115.83\pm11.88$\\
		& CTL & 28 & $26.75 \pm 7.32$ & $115.25\pm13.67$\\\midrule
		\multirow{2}{*}{ALL} &ASD & 68 & $25.90\pm6.95$ &	$109.03\pm13.31$\\
		& CTL & 68 & $27.93 \pm 5.87$ &$113.22\pm12.81$\\
		\bottomrule
	\end{tabularx}
	\caption[Demographics of the AIMS-MRI dataset.]{Demographics of the AIMS-MRI dataset.}
	\label{tab:demoMRCAIMS}
\end{table}

Structural MRI were obtained using Driven Equilibrium Single Pulse
Observation of T1 and T2 (DESPOT1, DESPOT2) (Deoni, et al., 2008) at
King’s College London and University of Cambridge, both with 3T GE
Medical Systems HDx scanners. Using multiple Spoilt Gradient Recall
(SPGR) acquisitions in the DESPOT1 sequence and Steady State Free
Procession (SSPF) acquisitions in the DESPOT2 sequence, with different
flip angles and repetition times, \ac{qT1} and
\ac{qT2} maps were calculated with a custom ImageJ plug-in package. Correction
of main and transmit magnetic field (B0 and B1) inhomogeneity effects
was performed during the estimation of T1 and T2.

For accurate registration to the standard stereotatic space of the
\ac{MNI}, a \ac{synT1} images were created based on the \ac{qT1}
maps (Ecker, et al., 2013; Ecker, et al., 2012; Lai, et al., 2012). The
\ac{synT1} images were then segmented using New Segment into \ac{GM} and
\ac{WM} maps, and normalized to the \ac{MNI} space using DARTEL in
SPM8 (Friston, et al., 2007), with modulation (preserve volume) to
retain information of regional/local \ac{GM} and \ac{WM} volumes, and smoothed
with a 3mm FWHM Gaussian Kernel to account for inter-subject
mis-registration. The \ac{synT1}, \ac{qT1} and \ac{qT2} maps were also registered to
the standard \ac{MNI} space using the same DARTEL flow fields, but without
modulation (preserve concentration) to retain information of
regional/local T1 contrast, T1 relaxation time, and T2 relaxation
times, and smoothed with a 3mm FWHM Gaussian kernel. Therefore, there
were five different modalities: \ac{qT1}, \ac{qT2}, \ac{synT1} map, \ac{GM} and \ac{WM} maps,
for each every participant, which allows us to observe the impact of our \ac{SWPCA} correction of site-related undesired variance on quantitative (\ac{qT1} and \ac{qT2}), simulated (\ac{synT1}) images and probability maps (\ac{GM} and \ac{WM}).
	
During the pre-processing of the images, several procedures targeted the reduction of inter-subject and inter-site geometric distortion, amongst them the correction of B0 and B1 field inhomogeneity effects and the registration to \ac{MNI} space. Many other algorithms have been proposed to help in this task. However, the study of their relative performance lies beyond the scope of this article. Following image registration, it was assumed that only the intensity of the maps was affected between sites.

\section{Positron Emission Tomography}
\subsection{ADNI-PET, Alzheimer's Disease Neuroimaging Initiative}\label{sec:adnipet}

Data used in the preparation of this article were obtained from the Alzheimer's Disease Neuroimaging Initiative (ADNI) database (adni.loni.usc.edu). The ADNI was launched in 2003 as a public-private partnership, led by Principal Investigator Michael W. Weiner, MD. The primary goal of ADNI has been to test whether serial MRI, PET, other biological markers, and clinical and neuropsychological assessment can be combined to measure the progression of MCI and AD. For up-to-date information, see \url{www.adni-info.org}.

Data used in the preparation of this article were obtained from the Alzheimer's Disease Neuroimaging Initiative (ADNI) database (adni.loni.usc.edu). In this work, the $^{18}F$-FDG PET images, used to estimate the metabolic activity of the brain, are used to generate and validate the simulated images. 95 PET images from AD affected subjects,  207 images from Mild Cognitive Impairment (MCI) affected subjects and 101 images from Normal Controls (NOR) have been used to construct the original $N=403$ set from which the simulation parameters will be obtained. 
\section{Single Photon Emission Computed Tomography}

\subsection{VDLN-HMPAO, Virgen de las Nieves}\label{sec:vdlnhmpao}
The database is built up of imaging studies of subjects following the protocol of an hospital-based service. First, the neurologist evaluated the cognitive function, and those patients with findings of memory loss or dementia were referred to the nuclear medicine department in the ``Virgen de las Nieves'' hospital (Granada, Spain), in order to acquire complementary screening information for diagnosis\footnote{Clinical information is unfortunately not available for privacy reasons, but only demographic  information}. Experienced physicians evaluated the images visually. The images were assessed using 4 different labels: \ac{CTL} for subjects without scintigraphic abnormalities and mild perfusion deficit (AD1), moderate deficit (AD2) and severe deficit (AD3), to distinguish between different levels of presence of hypo-perfusion patterns compatible with AD. In total, the database consists of $n=$97 subjects: 41 CTRL, 30 AD1, 22 AD2 and 4 AD3 (see table \ref{tab:bd} for demographic details). Since the patients are not pathologically confirmed, the subject's labels possesses some degree of uncertainty, as the pattern of hypo-perfusion may not reflect the underlying pathology of AD, nor the different classification of scans necessarily reflect the severity of the patients symptoms. However, when pathological information is available, visual assessments by experts have been shown to be very sensitive and specific labelling methods, in contrast to neuropsychological tests \cite{jobst_accurate_1998,dougall_systematic_2004}. Given that this is an inherent limitation of 'in vivo' studies, our working-assumption is that the labels are true, considering the subject label positive when belonging to any of the AD classes, and negative otherwise. 

\begin{table}
	\begin{center}
		\begin{tabular}{lccr}
			\hline
			\hline
			& $\#$samples & Sex(M/F)(\%) & $\mu$[range/$\sigma$] \\
			\hline \hline
			CTRL  & 41 & 32.95/12.19 & 71.51[46-85/7.99]\\
			AD1     & 29 & 10.97/18.29 &65.29[23-81/13.36]\\
			AD2     & 22 & 13.41/9.76  &65.73[46-86/8.25]\\
			AD3     & 4  & 0/2.43      &76[69-83/9.90]\\
			\hline \hline
		\end{tabular}
		\caption[Demographic details of the ADNI-PET dataset.]{Demographic details of the ADNI-PET dataset. CTRL = Normal Controls, AD 1 = possible AD, AD 2 = probable AD, AD 3 = certain AD. $\mu$ and $\sigma$ stands for population mean and standard deviation respectively.}
		\label{tab:bd}
	\end{center}
\end{table}

\subsection{VDLN-DAT, Virgen de las Nieves}\label{sec:vdlndat}
SPECT DATSCAN


73 CTL, 45 PD, 30 SWEDD. 

\subsection{VDLV-DAT, Virgen de la Victoria Hospital}\label{sec:vdlvdat}
The images were obtained after a period of between 3 and 4 hours after the intravenous injection of 185 MBq (5 mCi) of DaTSCAN, with prior thyroid blocking with Lugol's solution. The tomographic study (SPECT) with Ioflupane/FP-CIT-I-123 was performed using a General Electric gamma camera, Millennium model, equipped with a dual head and general purpose collimator. A 360-degree circular orbit was made around the cranium, at 3-degree intervals, 60 images with a duration of 35 seconds per interval, $128\times128$ matrix. Image reconstruction was carried out using filtered back-projection algorithms without attenuation correction \cite{Shepp82,Vardi1985}, application of a Hanning filter (frequency 0.7) and images were obtained with transaxial cuts, following the method proposed in \cite{Ramirez2009}. 

The images were interpreted by three Nuclear Medicine specialists, with masking of the clinical orientation. Visual assessment was established by exclusively considering the normal/abnormal criterion and after arriving at a consensus report between the three specialists, i.e. whether the FP-CIT SPECT allowed differentiation of a group of conditions with presynaptic involvement from others in which their integrity is assumed, without trying to assign them to different clinical groups within the set of pathological studies. A study was considered to be normal when bilateral, symmetrical uptake appeared in caudate and putamen nuclei, and abnormal when there were areas of qualitatively reduced uptake in any of the striatal structures. 

A total of 208 subjects (100 patients and 108 controls), randomly selected from the total studies performed in this center until December 2008 and referred to it because of a movement disorder, were included in the study. Mean age was 70.2 years (41-87) with a standard deviation of 10.2 years (a detailed description of the database can be found in \cite{Lozano2007}). Clinical diagnosis, a parameter used as `gold Standard' to establish the existence of PS, was made using the diagnostic criteria established previously, with an established minimum follow-up period of 18 months. Those patients who were receiving treatment with drugs that had known or suspected effect on the level of the dopaminergic transporters through direct competitive mechanism were excluded. Although PD is the most representative pathology of PS, there are other medical conditions which, though they differ clinically from this, are also expressed by this set of symptoms. Some of them are multisystem atrophy (MSA), progressive supra-nuclear palsy (PSP) and corticobasal degeneration (CBD), in which, unlike PD, as well as involvement of the presynaptic terminal, there is involvement at the post-synaptic level of the nigrostriatal pathway. 

\subsection{PPMI-DAT, Parkinson's Progression Markers Initiative}\label{sec:ppmi}
Data used in the preparation of this article were obtained from the Parkinson's Progression Markers Initiative (PPMI) database (\url{www.ppmi-info.org/data}). For up-to-date information on the study, visit \url{www.ppmi-info.org}.

The images in this database were imaged 4 + 0.5 hours after the injection of between 111 and 185 MBq of DaTSCAN. Subjects were also pretreated with saturated iodine solution (10 drops in water) or perchlorate (1000 mg) prior to the injection. All subjects had a supplied $^{57}$Co line marker affixed along the canthomeatal line, which will facilitate subsequent image processing and allow the core lab to accurately distinguish left and right in the face of multiple image file transfers. These markers are only evident in the $^{57}$Co window and hence do not contaminate the $^{123}$I-DaTSCAN brain data \cite{PPMI,Inititative2010}. 

111 CTL, 32 SWEDD and 158 PD

Raw projection data are acquired into a $128 \times 128$ matrix stepping each 3 degrees for a total of 120 projection into two 20\% symmetric photopeak windows centered on 159 KeV and 122 KeV with a total scan duration of approximately 30 - 45 minutes. Other scan parameters (collimation, acquisition mode, etc.) are selected for each site. The images of both the subject's data and the cobalt striatal phantom are reconstructed and attenuation corrected, implementing either filtered back-projection or an iterative reconstruction algorithm using standardized approaches \cite{Inititative2010}. After the processing, the database contains $289$ spatially normalized images, $114$ from Normal Control subjects and $175$ from PD patients, and of a $91\times109\times91$ size. 
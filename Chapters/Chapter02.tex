%*****************************************
\chapter{Medical Background}\label{ch:medicine}
%*****************************************

First of all, it is interesting to provide some medical background about the diseases that we have applied or methodology to. This is the case of \ac{AD}, \ac{PKS} and \ac{ASD}. In this chapter, we will explore what we currently know about causes, symptoms and particularities of these diseases, and how these can be identified using neuroimaging. 

\section{Alzheimer's Disease}
\ac{AD} is whatever.. 

Wikipedia: The cause of Alzheimer's disease is poorly understood [1]. About 70\% of the risk is believed to be genetic with many genes usually involved.[6] Other risk factors include a history of head injuries, depression, or hypertension.[1] The disease process is associated with plaques and tangles in the brain.[6] A probable diagnosis is based on the history of the illness and cognitive testing with medical imaging and blood tests to rule out other possible causes.[7] Initial symptoms are often mistaken for normal ageing.[1] Examination of brain tissue is needed for a definite diagnosis.[6] Mental and physical exercise, and avoiding obesity may decrease the risk of AD.[6] There are no medications or supplements that decrease risk.[8]
\subsection{Diagnosis}
TEsts

\section{Parkinsonism}
\subsection{Parkinson's Disease}
\subsection{Extrapyramidal Symptoms}
\subsection{Diagnosis}

\section{Autism Spectrum Disorder}
\subsection{Diagnosis}

